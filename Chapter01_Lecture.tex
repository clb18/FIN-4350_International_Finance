% Options for packages loaded elsewhere
% Options for packages loaded elsewhere
\PassOptionsToPackage{unicode}{hyperref}
\PassOptionsToPackage{hyphens}{url}
\PassOptionsToPackage{dvipsnames,svgnames,x11names}{xcolor}
%
\documentclass[
  letterpaper,
  DIV=11,
  numbers=noendperiod]{scrartcl}
\usepackage{xcolor}
\usepackage[margin=1in]{geometry}
\usepackage{amsmath,amssymb}
\setcounter{secnumdepth}{5}
\usepackage{iftex}
\ifPDFTeX
  \usepackage[T1]{fontenc}
  \usepackage[utf8]{inputenc}
  \usepackage{textcomp} % provide euro and other symbols
\else % if luatex or xetex
  \usepackage{unicode-math} % this also loads fontspec
  \defaultfontfeatures{Scale=MatchLowercase}
  \defaultfontfeatures[\rmfamily]{Ligatures=TeX,Scale=1}
\fi
\usepackage{lmodern}
\ifPDFTeX\else
  % xetex/luatex font selection
\fi
% Use upquote if available, for straight quotes in verbatim environments
\IfFileExists{upquote.sty}{\usepackage{upquote}}{}
\IfFileExists{microtype.sty}{% use microtype if available
  \usepackage[]{microtype}
  \UseMicrotypeSet[protrusion]{basicmath} % disable protrusion for tt fonts
}{}
\makeatletter
\@ifundefined{KOMAClassName}{% if non-KOMA class
  \IfFileExists{parskip.sty}{%
    \usepackage{parskip}
  }{% else
    \setlength{\parindent}{0pt}
    \setlength{\parskip}{6pt plus 2pt minus 1pt}}
}{% if KOMA class
  \KOMAoptions{parskip=half}}
\makeatother
% Make \paragraph and \subparagraph free-standing
\makeatletter
\ifx\paragraph\undefined\else
  \let\oldparagraph\paragraph
  \renewcommand{\paragraph}{
    \@ifstar
      \xxxParagraphStar
      \xxxParagraphNoStar
  }
  \newcommand{\xxxParagraphStar}[1]{\oldparagraph*{#1}\mbox{}}
  \newcommand{\xxxParagraphNoStar}[1]{\oldparagraph{#1}\mbox{}}
\fi
\ifx\subparagraph\undefined\else
  \let\oldsubparagraph\subparagraph
  \renewcommand{\subparagraph}{
    \@ifstar
      \xxxSubParagraphStar
      \xxxSubParagraphNoStar
  }
  \newcommand{\xxxSubParagraphStar}[1]{\oldsubparagraph*{#1}\mbox{}}
  \newcommand{\xxxSubParagraphNoStar}[1]{\oldsubparagraph{#1}\mbox{}}
\fi
\makeatother

\usepackage{color}
\usepackage{fancyvrb}
\newcommand{\VerbBar}{|}
\newcommand{\VERB}{\Verb[commandchars=\\\{\}]}
\DefineVerbatimEnvironment{Highlighting}{Verbatim}{commandchars=\\\{\}}
% Add ',fontsize=\small' for more characters per line
\usepackage{framed}
\definecolor{shadecolor}{RGB}{241,243,245}
\newenvironment{Shaded}{\begin{snugshade}}{\end{snugshade}}
\newcommand{\AlertTok}[1]{\textcolor[rgb]{0.68,0.00,0.00}{#1}}
\newcommand{\AnnotationTok}[1]{\textcolor[rgb]{0.37,0.37,0.37}{#1}}
\newcommand{\AttributeTok}[1]{\textcolor[rgb]{0.40,0.45,0.13}{#1}}
\newcommand{\BaseNTok}[1]{\textcolor[rgb]{0.68,0.00,0.00}{#1}}
\newcommand{\BuiltInTok}[1]{\textcolor[rgb]{0.00,0.23,0.31}{#1}}
\newcommand{\CharTok}[1]{\textcolor[rgb]{0.13,0.47,0.30}{#1}}
\newcommand{\CommentTok}[1]{\textcolor[rgb]{0.37,0.37,0.37}{#1}}
\newcommand{\CommentVarTok}[1]{\textcolor[rgb]{0.37,0.37,0.37}{\textit{#1}}}
\newcommand{\ConstantTok}[1]{\textcolor[rgb]{0.56,0.35,0.01}{#1}}
\newcommand{\ControlFlowTok}[1]{\textcolor[rgb]{0.00,0.23,0.31}{\textbf{#1}}}
\newcommand{\DataTypeTok}[1]{\textcolor[rgb]{0.68,0.00,0.00}{#1}}
\newcommand{\DecValTok}[1]{\textcolor[rgb]{0.68,0.00,0.00}{#1}}
\newcommand{\DocumentationTok}[1]{\textcolor[rgb]{0.37,0.37,0.37}{\textit{#1}}}
\newcommand{\ErrorTok}[1]{\textcolor[rgb]{0.68,0.00,0.00}{#1}}
\newcommand{\ExtensionTok}[1]{\textcolor[rgb]{0.00,0.23,0.31}{#1}}
\newcommand{\FloatTok}[1]{\textcolor[rgb]{0.68,0.00,0.00}{#1}}
\newcommand{\FunctionTok}[1]{\textcolor[rgb]{0.28,0.35,0.67}{#1}}
\newcommand{\ImportTok}[1]{\textcolor[rgb]{0.00,0.46,0.62}{#1}}
\newcommand{\InformationTok}[1]{\textcolor[rgb]{0.37,0.37,0.37}{#1}}
\newcommand{\KeywordTok}[1]{\textcolor[rgb]{0.00,0.23,0.31}{\textbf{#1}}}
\newcommand{\NormalTok}[1]{\textcolor[rgb]{0.00,0.23,0.31}{#1}}
\newcommand{\OperatorTok}[1]{\textcolor[rgb]{0.37,0.37,0.37}{#1}}
\newcommand{\OtherTok}[1]{\textcolor[rgb]{0.00,0.23,0.31}{#1}}
\newcommand{\PreprocessorTok}[1]{\textcolor[rgb]{0.68,0.00,0.00}{#1}}
\newcommand{\RegionMarkerTok}[1]{\textcolor[rgb]{0.00,0.23,0.31}{#1}}
\newcommand{\SpecialCharTok}[1]{\textcolor[rgb]{0.37,0.37,0.37}{#1}}
\newcommand{\SpecialStringTok}[1]{\textcolor[rgb]{0.13,0.47,0.30}{#1}}
\newcommand{\StringTok}[1]{\textcolor[rgb]{0.13,0.47,0.30}{#1}}
\newcommand{\VariableTok}[1]{\textcolor[rgb]{0.07,0.07,0.07}{#1}}
\newcommand{\VerbatimStringTok}[1]{\textcolor[rgb]{0.13,0.47,0.30}{#1}}
\newcommand{\WarningTok}[1]{\textcolor[rgb]{0.37,0.37,0.37}{\textit{#1}}}

\usepackage{longtable,booktabs,array}
\usepackage{calc} % for calculating minipage widths
% Correct order of tables after \paragraph or \subparagraph
\usepackage{etoolbox}
\makeatletter
\patchcmd\longtable{\par}{\if@noskipsec\mbox{}\fi\par}{}{}
\makeatother
% Allow footnotes in longtable head/foot
\IfFileExists{footnotehyper.sty}{\usepackage{footnotehyper}}{\usepackage{footnote}}
\makesavenoteenv{longtable}
\usepackage{graphicx}
\makeatletter
\newsavebox\pandoc@box
\newcommand*\pandocbounded[1]{% scales image to fit in text height/width
  \sbox\pandoc@box{#1}%
  \Gscale@div\@tempa{\textheight}{\dimexpr\ht\pandoc@box+\dp\pandoc@box\relax}%
  \Gscale@div\@tempb{\linewidth}{\wd\pandoc@box}%
  \ifdim\@tempb\p@<\@tempa\p@\let\@tempa\@tempb\fi% select the smaller of both
  \ifdim\@tempa\p@<\p@\scalebox{\@tempa}{\usebox\pandoc@box}%
  \else\usebox{\pandoc@box}%
  \fi%
}
% Set default figure placement to htbp
\def\fps@figure{htbp}
\makeatother





\setlength{\emergencystretch}{3em} % prevent overfull lines

\providecommand{\tightlist}{%
  \setlength{\itemsep}{0pt}\setlength{\parskip}{0pt}}



 


\KOMAoption{captions}{tableheading}
\makeatletter
\@ifpackageloaded{caption}{}{\usepackage{caption}}
\AtBeginDocument{%
\ifdefined\contentsname
  \renewcommand*\contentsname{Table of contents}
\else
  \newcommand\contentsname{Table of contents}
\fi
\ifdefined\listfigurename
  \renewcommand*\listfigurename{List of Figures}
\else
  \newcommand\listfigurename{List of Figures}
\fi
\ifdefined\listtablename
  \renewcommand*\listtablename{List of Tables}
\else
  \newcommand\listtablename{List of Tables}
\fi
\ifdefined\figurename
  \renewcommand*\figurename{Figure}
\else
  \newcommand\figurename{Figure}
\fi
\ifdefined\tablename
  \renewcommand*\tablename{Table}
\else
  \newcommand\tablename{Table}
\fi
}
\@ifpackageloaded{float}{}{\usepackage{float}}
\floatstyle{ruled}
\@ifundefined{c@chapter}{\newfloat{codelisting}{h}{lop}}{\newfloat{codelisting}{h}{lop}[chapter]}
\floatname{codelisting}{Listing}
\newcommand*\listoflistings{\listof{codelisting}{List of Listings}}
\makeatother
\makeatletter
\makeatother
\makeatletter
\@ifpackageloaded{caption}{}{\usepackage{caption}}
\@ifpackageloaded{subcaption}{}{\usepackage{subcaption}}
\makeatother
\usepackage{bookmark}
\IfFileExists{xurl.sty}{\usepackage{xurl}}{} % add URL line breaks if available
\urlstyle{same}
\hypersetup{
  pdftitle={International Finance - Chapter 1},
  pdfauthor={Prof.~Barker},
  colorlinks=true,
  linkcolor={blue},
  filecolor={Maroon},
  citecolor={Blue},
  urlcolor={Blue},
  pdfcreator={LaTeX via pandoc}}


\title{International Finance - Chapter 1}
\usepackage{etoolbox}
\makeatletter
\providecommand{\subtitle}[1]{% add subtitle to \maketitle
  \apptocmd{\@title}{\par {\large #1 \par}}{}{}
}
\makeatother
\subtitle{Multinational Financial Management: Opportunities and
Challenges}
\author{Prof.~Barker}
\date{2025-12-24}
\begin{document}
\maketitle

\renewcommand*\contentsname{Table of contents}
{
\hypersetup{linkcolor=}
\setcounter{tocdepth}{3}
\tableofcontents
}

\section{Introduction and Opening}\label{introduction-and-opening}

\subsection{Welcome and Context
Setting}\label{welcome-and-context-setting}

Today we begin our journey into the fascinating world of international
finance by exploring what makes multinational financial management both
uniquely challenging and remarkably rewarding.

\textbf{Opening Question to Class:} ``How many of you have used a
product today that was made by a company operating in multiple
countries?''

This simple exercise demonstrates that we live in a deeply
interconnected global economy. The phone in your pocket, the coffee you
drank this morning, the streaming service you use - all likely involve
multinational enterprises (MNEs) managing complex financial operations
across borders.

\subsection{Chapter Roadmap}\label{chapter-roadmap}

Today we'll cover four major learning objectives:

\begin{enumerate}
\def\labelenumi{\arabic{enumi}.}
\tightlist
\item
  \textbf{The Global Financial Marketplace} - Understanding the players,
  institutions, and linkages
\item
  \textbf{Theory of Comparative Advantage} - Why international trade and
  investment make economic sense
\item
  \textbf{What Makes International Finance Different} - The unique
  challenges MNEs face
\item
  \textbf{The Globalization Process} - How companies evolve from
  domestic to truly global operations
\end{enumerate}

\begin{center}\rule{0.5\linewidth}{0.5pt}\end{center}

\section{The Global Financial
Marketplace}\label{the-global-financial-marketplace}

\subsection{The Playing Field}\label{the-playing-field}

Think of the global financial marketplace as a complex ecosystem with
three fundamental components:

\subsubsection{Securities (Financial
Assets)}\label{securities-financial-assets}

\begin{itemize}
\tightlist
\item
  At the foundation are government debt securities - US Treasury Bonds,
  UK gilts, German bunds
\item
  These ``risk-free'' securities form the benchmark for pricing all
  other assets
\item
  Built upon this foundation: corporate bonds, equities (stocks), bank
  loans
\item
  More recently: derivatives - financial instruments whose value derives
  from underlying assets
\end{itemize}

\textbf{Critical Point:} The health of the global financial system
depends on the quality of these securities. When securities fail
(remember the 2008 financial crisis with mortgage-backed securities),
the entire system suffers.

\subsubsection{Institutions}\label{institutions}

\begin{itemize}
\tightlist
\item
  \textbf{Central Banks}: Create and control money supply (Federal
  Reserve, European Central Bank, Bank of Japan)
\item
  \textbf{Commercial Banks}: Take deposits and extend loans both
  domestically and internationally
\item
  \textbf{Investment Banks \& Financial Institutions}: Trade securities,
  facilitate capital flows, create financial products
\end{itemize}

\subsubsection{Linkages - The Interbank
Market}\label{linkages---the-interbank-market}

\begin{itemize}
\tightlist
\item
  This is the ``plumbing'' of the global financial system
\item
  Uses CURRENCY as the medium of exchange
\item
  Historically centered on LIBOR (London Interbank Offered Rate)
\item
  \textbf{Important note:} LIBOR is being phased out and replaced (we'll
  study this in Chapter 8)
\end{itemize}

\subsection{The Market for Currencies}\label{the-market-for-currencies}

\textbf{Key Concept:} A foreign exchange rate is simply the price of one
currency in terms of another.

\textbf{Example:}

\begin{verbatim}
EUR 1.00 = USD 1.1274
or
USD 1.1274 = EUR 1.00
\end{verbatim}

\subsubsection{Important Terminology
Discussion}\label{important-terminology-discussion}

Let me address something that often confuses students - currency
quotation conventions. There are different ways to express the same
exchange rate:

\begin{itemize}
\tightlist
\item
  \textbf{Traditional notation}: \$1.1274/€ (read as ``1.1274 dollars
  per euro'')
\item
  \textbf{ISO code format}: USD1.1274 = EUR1.00
\item
  \textbf{Financial press variations}:

  \begin{itemize}
  \tightlist
  \item
    EUR/USD 1.1274
  \item
    EUR-USD 1.1274
  \item
    EURUSD 1.1274
  \end{itemize}
\end{itemize}

All mean the same thing! The key is understanding which currency is the
``base'' (the one that equals 1.00) and which is the ``quote'' (the
price).

\textbf{Convention to Remember:}

\begin{itemize}
\tightlist
\item
  EUR, GBP (British pound), and some Commonwealth currencies are
  typically quoted as ``dollars per foreign currency''
\item
  Most other currencies quote as ``foreign currency per dollar''
\end{itemize}

\textbf{Why does this matter?} Because when you calculate percentage
changes or returns, you need to know which currency is in the numerator
and which is in the denominator.

\subsection{Calculating Percentage Changes in Exchange
Rates}\label{calculating-percentage-changes-in-exchange-rates}

\textbf{This is critical - pay close attention.}

Let's work through the Argentine peso example from your textbook:

\textbf{Scenario:} December 16, 2015 - Argentina lifts currency controls

\begin{itemize}
\tightlist
\item
  Beginning rate: ARG 9.7908 = USD 1.00
\item
  Ending rate: ARG 13.6160 = USD 1.00
\end{itemize}

\textbf{Question:} What happened to the peso's value?

\subsubsection{Method 1 - Foreign Currency Terms (Foreign currency per
USD)}\label{method-1---foreign-currency-terms-foreign-currency-per-usd}

\begin{Shaded}
\begin{Highlighting}[]
\CommentTok{\# Given exchange rates}
\NormalTok{begin\_rate }\OtherTok{\textless{}{-}} \FloatTok{9.7908}  \CommentTok{\# ARG per USD}
\NormalTok{end\_rate }\OtherTok{\textless{}{-}} \FloatTok{13.6160}   \CommentTok{\# ARG per USD}

\CommentTok{\# Calculate percentage change}
\NormalTok{pct\_change\_method1 }\OtherTok{\textless{}{-}}\NormalTok{ ((begin\_rate }\SpecialCharTok{{-}}\NormalTok{ end\_rate) }\SpecialCharTok{/}\NormalTok{ end\_rate) }\SpecialCharTok{*} \DecValTok{100}

\FunctionTok{cat}\NormalTok{(}\StringTok{"Method 1 Calculation:}\SpecialCharTok{\textbackslash{}n}\StringTok{"}\NormalTok{)}
\end{Highlighting}
\end{Shaded}

\begin{verbatim}
Method 1 Calculation:
\end{verbatim}

\begin{Shaded}
\begin{Highlighting}[]
\FunctionTok{cat}\NormalTok{(}\StringTok{"Beginning rate: ARG"}\NormalTok{, begin\_rate, }\StringTok{"= USD 1.00}\SpecialCharTok{\textbackslash{}n}\StringTok{"}\NormalTok{)}
\end{Highlighting}
\end{Shaded}

\begin{verbatim}
Beginning rate: ARG 9.7908 = USD 1.00
\end{verbatim}

\begin{Shaded}
\begin{Highlighting}[]
\FunctionTok{cat}\NormalTok{(}\StringTok{"Ending rate: ARG"}\NormalTok{, end\_rate, }\StringTok{"= USD 1.00}\SpecialCharTok{\textbackslash{}n}\StringTok{"}\NormalTok{)}
\end{Highlighting}
\end{Shaded}

\begin{verbatim}
Ending rate: ARG 13.616 = USD 1.00
\end{verbatim}

\begin{Shaded}
\begin{Highlighting}[]
\FunctionTok{cat}\NormalTok{(}\StringTok{"Percentage change:"}\NormalTok{, }\FunctionTok{round}\NormalTok{(pct\_change\_method1, }\DecValTok{2}\NormalTok{), }\StringTok{"\%}\SpecialCharTok{\textbackslash{}n}\StringTok{"}\NormalTok{)}
\end{Highlighting}
\end{Shaded}

\begin{verbatim}
Percentage change: -28.09 %
\end{verbatim}

\textbf{Interpretation:} The peso FELL 28\% in value. Notice:

\begin{itemize}
\tightlist
\item
  It takes MORE pesos to buy one dollar (went from 9.79 to 13.62)
\item
  The calculation yielded a NEGATIVE number
\item
  Both indicate depreciation
\end{itemize}

\subsubsection{Method 2 - Home Currency Terms (USD per foreign
currency)}\label{method-2---home-currency-terms-usd-per-foreign-currency}

First, we need to flip the rates:

\begin{Shaded}
\begin{Highlighting}[]
\CommentTok{\# Convert to USD per ARG}
\NormalTok{begin\_rate\_usd }\OtherTok{\textless{}{-}} \DecValTok{1} \SpecialCharTok{/}\NormalTok{ begin\_rate}
\NormalTok{end\_rate\_usd }\OtherTok{\textless{}{-}} \DecValTok{1} \SpecialCharTok{/}\NormalTok{ end\_rate}

\CommentTok{\# Calculate percentage change}
\NormalTok{pct\_change\_method2 }\OtherTok{\textless{}{-}}\NormalTok{ ((end\_rate\_usd }\SpecialCharTok{{-}}\NormalTok{ begin\_rate\_usd) }\SpecialCharTok{/}\NormalTok{ begin\_rate\_usd) }\SpecialCharTok{*} \DecValTok{100}

\FunctionTok{cat}\NormalTok{(}\StringTok{"Method 2 Calculation:}\SpecialCharTok{\textbackslash{}n}\StringTok{"}\NormalTok{)}
\end{Highlighting}
\end{Shaded}

\begin{verbatim}
Method 2 Calculation:
\end{verbatim}

\begin{Shaded}
\begin{Highlighting}[]
\FunctionTok{cat}\NormalTok{(}\StringTok{"Beginning rate: USD"}\NormalTok{, }\FunctionTok{round}\NormalTok{(begin\_rate\_usd, }\DecValTok{5}\NormalTok{), }\StringTok{"= ARG 1.00}\SpecialCharTok{\textbackslash{}n}\StringTok{"}\NormalTok{)}
\end{Highlighting}
\end{Shaded}

\begin{verbatim}
Beginning rate: USD 0.10214 = ARG 1.00
\end{verbatim}

\begin{Shaded}
\begin{Highlighting}[]
\FunctionTok{cat}\NormalTok{(}\StringTok{"Ending rate: USD"}\NormalTok{, }\FunctionTok{round}\NormalTok{(end\_rate\_usd, }\DecValTok{5}\NormalTok{), }\StringTok{"= ARG 1.00}\SpecialCharTok{\textbackslash{}n}\StringTok{"}\NormalTok{)}
\end{Highlighting}
\end{Shaded}

\begin{verbatim}
Ending rate: USD 0.07344 = ARG 1.00
\end{verbatim}

\begin{Shaded}
\begin{Highlighting}[]
\FunctionTok{cat}\NormalTok{(}\StringTok{"Percentage change:"}\NormalTok{, }\FunctionTok{round}\NormalTok{(pct\_change\_method2, }\DecValTok{2}\NormalTok{), }\StringTok{"\%}\SpecialCharTok{\textbackslash{}n}\StringTok{"}\NormalTok{)}
\end{Highlighting}
\end{Shaded}

\begin{verbatim}
Percentage change: -28.09 %
\end{verbatim}

\textbf{Same answer!} The peso fell 28\%.

\textbf{Teaching Point:} Many students find Method 2 more intuitive
because it follows the standard ``end minus beginning over beginning''
formula. However, both methods MUST give the same answer.

\subsubsection{Practice Problem}\label{practice-problem}

If the Mexican peso changes from MXN 16.00 = USD 1.00 to MXN 20.00 = USD
1.00, what is the percentage change?

\begin{Shaded}
\begin{Highlighting}[]
\CommentTok{\# Mexican peso example}
\NormalTok{mxn\_begin }\OtherTok{\textless{}{-}} \FloatTok{16.00}
\NormalTok{mxn\_end }\OtherTok{\textless{}{-}} \FloatTok{20.00}

\CommentTok{\# Calculate percentage change}
\NormalTok{mxn\_pct\_change }\OtherTok{\textless{}{-}}\NormalTok{ ((mxn\_begin }\SpecialCharTok{{-}}\NormalTok{ mxn\_end) }\SpecialCharTok{/}\NormalTok{ mxn\_end) }\SpecialCharTok{*} \DecValTok{100}

\FunctionTok{cat}\NormalTok{(}\StringTok{"Mexican Peso Change:}\SpecialCharTok{\textbackslash{}n}\StringTok{"}\NormalTok{)}
\end{Highlighting}
\end{Shaded}

\begin{verbatim}
Mexican Peso Change:
\end{verbatim}

\begin{Shaded}
\begin{Highlighting}[]
\FunctionTok{cat}\NormalTok{(}\StringTok{"Answer:"}\NormalTok{, }\FunctionTok{round}\NormalTok{(mxn\_pct\_change, }\DecValTok{2}\NormalTok{), }\StringTok{"\%}\SpecialCharTok{\textbackslash{}n}\StringTok{"}\NormalTok{)}
\end{Highlighting}
\end{Shaded}

\begin{verbatim}
Answer: -20 %
\end{verbatim}

\subsection{Eurocurrencies and the Eurocurrency
Market}\label{eurocurrencies-and-the-eurocurrency-market}

\textbf{Critical Definition:} Eurocurrencies are deposits of a currency
held in banks OUTSIDE that currency's home country.

\textbf{Examples:}

\begin{itemize}
\tightlist
\item
  \textbf{Eurodollars}: US dollars deposited in a London bank
\item
  \textbf{Euroyen}: Japanese yen deposited in a Singapore bank
\item
  \textbf{Eurosterling}: British pounds deposited in a Paris bank
\end{itemize}

\textbf{Important:} The ``euro'' prefix has NOTHING to do with the
European currency. This terminology predates the euro.

\subsubsection{Historical Origin (Fascinating
backstory)}\label{historical-origin-fascinating-backstory}

\begin{itemize}
\tightlist
\item
  Born after WWII
\item
  Soviet Union and Eastern European countries held US dollars
\item
  Feared US government might freeze these dollar assets
\item
  Deposited dollars in European banks instead (particularly in London)
\item
  The market grew from there
\end{itemize}

\subsubsection{Why Does This Market Exist and
Thrive?}\label{why-does-this-market-exist-and-thrive}

\begin{enumerate}
\def\labelenumi{\arabic{enumi}.}
\tightlist
\item
  \textbf{Regulatory Arbitrage}: Fewer regulations than domestic banking

  \begin{itemize}
  \tightlist
  \item
    No reserve requirements
  \item
    No deposit insurance costs (like FDIC in the US)
  \item
    Less bureaucratic overhead
  \end{itemize}
\item
  \textbf{Narrow Spreads}: The difference between deposit and loan rates
  is often less than 1\%

  \begin{itemize}
  \tightlist
  \item
    Why? Wholesale market (large transactions, \$500,000+)
  \item
    Creditworthy borrowers (corporations, governments)
  \item
    Lower overhead costs
  \end{itemize}
\item
  \textbf{Two Primary Functions}:

  \begin{itemize}
  \tightlist
  \item
    \textbf{Money market device}: Corporations park excess cash, earning
    higher interest than domestic alternatives
  \item
    \textbf{Lending source}: Companies borrow for working capital, trade
    finance
  \end{itemize}
\end{enumerate}

\textbf{LIBOR Connection:} The London Interbank Offered Rate (LIBOR) is
the benchmark interest rate for the eurocurrency market. It represents
the rate at which banks lend to each other in London.

\textbf{Current Developments:} LIBOR is being replaced due to
manipulation scandals. We'll explore this transition in detail in
Chapter 8.

\begin{center}\rule{0.5\linewidth}{0.5pt}\end{center}

\section{Theory of Comparative
Advantage}\label{theory-of-comparative-advantage}

\subsection{Foundational Economic
Theory}\label{foundational-economic-theory}

\textbf{Question to class:} ``Why do countries trade with each other?''

The theory of comparative advantage, developed by David Ricardo in 1817
(building on Adam Smith's work from 1776), provides the intellectual
foundation for understanding international trade and investment.

\subsection{Core Concepts}\label{core-concepts}

\subsubsection{Absolute Advantage (Adam
Smith)}\label{absolute-advantage-adam-smith}

\begin{itemize}
\tightlist
\item
  A country should specialize in producing goods it can make most
  efficiently
\item
  Trade these for goods other countries make more efficiently
\item
  Result: More total production, lower prices, higher quality of life
\end{itemize}

\subsubsection{Comparative Advantage (David
Ricardo)}\label{comparative-advantage-david-ricardo}

\begin{itemize}
\tightlist
\item
  Even if one country has absolute advantage in producing EVERYTHING,
  both countries benefit from specialization
\item
  Each country should specialize in what it does RELATIVELY better
\item
  The key word is ``comparative'' - relative, not absolute
\end{itemize}

\subsubsection{Simple Example}\label{simple-example}

Imagine two countries - the US and China:

\begin{itemize}
\tightlist
\item
  US: Can produce either 100 computers OR 100 bushels of wheat with the
  same resources
\item
  China: Can produce either 50 computers OR 150 bushels of wheat with
  the same resources
\end{itemize}

Even though the US has absolute advantage in computers, China might
still have comparative advantage in wheat production if the opportunity
costs differ appropriately.

\subsection{Modern Reality vs.~Theory}\label{modern-reality-vs.-theory}

The textbook presents important limitations to pure comparative
advantage theory:

\subsubsection{Government Interference}\label{government-interference}

\begin{itemize}
\tightlist
\item
  Tariffs and quotas
\item
  Subsidies to domestic industries
\item
  Political motivations (food security, defense industries, employment
  goals)
\end{itemize}

\textbf{Example:} Many countries protect their agricultural sectors even
when they're not the most efficient producers, because food security is
considered a national security issue.

\subsubsection{Factor Mobility}\label{factor-mobility}

\begin{itemize}
\tightlist
\item
  Classical theory assumed factors of production (land, labor, capital)
  stayed within countries
\item
  Modern reality: Capital flows freely across borders
\item
  Technology transfers rapidly
\item
  Even labor moves (though less freely than capital)
\end{itemize}

\subsubsection{More Complex Factors of
Production}\label{more-complex-factors-of-production}

Modern location decisions consider:

\begin{itemize}
\tightlist
\item
  Managerial expertise
\item
  Legal and regulatory environment
\item
  R\&D capabilities
\item
  Education levels of workforce
\item
  Infrastructure quality
\item
  Access to capital markets
\item
  Tax differentials
\item
  Political stability
\end{itemize}

\subsubsection{Comparative Advantage Shifts Over
Time}\label{comparative-advantage-shifts-over-time}

\textbf{Powerful Example from Textbook:}

Cotton textile production comparative advantage has shifted over 150
years:

\begin{itemize}
\tightlist
\item
  United Kingdom (1800s) → United States (early 1900s) → Japan
  (mid-1900s) → Hong Kong → Taiwan → China (present)
\end{itemize}

This illustrates how comparative advantage is dynamic, not static.

\subsection{Relevance Today}\label{relevance-today}

Despite limitations, comparative advantage remains highly relevant:

\textbf{Modern Application:} Service outsourcing and the global supply
chain

\begin{itemize}
\tightlist
\item
  Financial back offices in Manila (Philippines)
\item
  IT engineering in Hungary and India
\item
  Manufacturing in Vietnam and Bangladesh
\item
  Design and innovation in Silicon Valley
\end{itemize}

\textbf{Key Insight:} Modern comparative advantage is based on:

\begin{itemize}
\tightlist
\item
  Labor skills and costs
\item
  Access to capital
\item
  Technology and innovation capacity
\item
  Telecommunications infrastructure
\end{itemize}

\textbf{Question for Discussion:} ``Can you think of services or
products where the US maintains comparative advantage?''

(Expected answers: innovation, higher education, entertainment, certain
high-tech manufacturing)

\begin{center}\rule{0.5\linewidth}{0.5pt}\end{center}

\section{What Makes International Finance Different? (15
minutes)}\label{what-makes-international-finance-different-15-minutes}

\subsection{Framework for Understanding
Differences}\label{framework-for-understanding-differences}

International financial management differs from domestic finance in six
key dimensions:

\subsection{Culture, History, and Institutional
Differences}\label{culture-history-and-institutional-differences}

\subsubsection{Cultural Challenges}\label{cultural-challenges}

\begin{itemize}
\tightlist
\item
  Business practices vary enormously
\item
  Contract interpretation differences
\item
  Relationship vs.~transaction orientation
\item
  Time orientation (short-term vs.~long-term)
\end{itemize}

\textbf{Example:} In many Asian business cultures, building
relationships (guanxi in China) is prerequisite to doing business. In
the US, contracts often precede relationships.

\subsubsection{Historical Context}\label{historical-context}

\begin{itemize}
\tightlist
\item
  Legal systems differ (common law vs.~civil law vs.~religious law)
\item
  Accounting standards vary (though IFRS is harmonizing some)
\item
  Property rights protection differs dramatically
\end{itemize}

\subsubsection{Institutional Frameworks}\label{institutional-frameworks}

\begin{itemize}
\tightlist
\item
  Banking system structures
\item
  Capital market development levels
\item
  Government roles in economy
\item
  Corruption and governance quality
\end{itemize}

\textbf{Real World Impact:} A company expanding to a new country must
understand these differences. What works in New York may fail in Mumbai
or Moscow - not because the financial principles are different, but
because the operating environment is different.

\subsection{Corporate Governance
Differences}\label{corporate-governance-differences}

\textbf{Critical Point:} Not all countries have the same ownership
structures or governance models.

\subsubsection{Three Primary Models}\label{three-primary-models}

\textbf{1. Anglo-American Model:}

\begin{itemize}
\tightlist
\item
  Dispersed shareholders
\item
  Active stock markets
\item
  Strong shareholder rights
\item
  Goal: Maximize shareholder value
\item
  Examples: US, UK, Canada, Australia
\end{itemize}

\textbf{2. Continental European Model:}

\begin{itemize}
\tightlist
\item
  Concentrated ownership (families, banks, other corporations)
\item
  Stakeholder orientation (employees, community, shareholders)
\item
  Bank-centered finance
\item
  Examples: Germany, France, Italy
\end{itemize}

\textbf{3. Asian Model:}

\begin{itemize}
\tightlist
\item
  Often family-controlled (even in public companies)
\item
  Keiretsu in Japan, Chaebol in Korea
\item
  Network relationships important
\item
  State ownership common (especially China)
\end{itemize}

\textbf{Why This Matters:}

\begin{itemize}
\tightlist
\item
  Affects access to capital
\item
  Influences financial decisions
\item
  Determines accountability structures
\item
  Shapes strategic objectives
\end{itemize}

\subsection{Foreign Exchange Risk}\label{foreign-exchange-risk}

\textbf{This is THE defining characteristic of international finance.}

\begin{itemize}
\tightlist
\item
  \textbf{Domestic firm:} All transactions in one currency
\item
  \textbf{International trading firm:} Some transactions in foreign
  currencies (import/export)
\item
  \textbf{Multinational firm:} Extensive foreign currency exposure
  through:

  \begin{itemize}
  \tightlist
  \item
    Foreign sales and revenues
  \item
    Foreign costs and expenses\\
  \item
    Foreign assets and liabilities
  \item
    Foreign subsidiary operations
  \end{itemize}
\end{itemize}

\subsubsection{Types of Foreign Exchange
Exposure}\label{types-of-foreign-exchange-exposure}

\textbf{1. Transaction Exposure:}

\begin{itemize}
\tightlist
\item
  Committed but not yet settled foreign currency transactions
\item
  Example: US company sells to German customer, invoice for €1 million
  due in 90 days
\item
  Risk: Euro value could fall during 90 days
\end{itemize}

\textbf{2. Translation (Accounting) Exposure:}

\begin{itemize}
\tightlist
\item
  Converting foreign subsidiary financial statements to home currency
\item
  Even if foreign operations are stable, exchange rate changes affect
  reported results
\end{itemize}

\textbf{3. Economic (Operating) Exposure:}

\begin{itemize}
\tightlist
\item
  Long-term competitiveness effects from exchange rate changes
\item
  Hardest to measure but potentially most significant
\end{itemize}

\subsubsection{Classroom Example: Ganado
Corporation}\label{classroom-example-ganado-corporation}

Let's create a simple consolidation example:

\begin{Shaded}
\begin{Highlighting}[]
\CommentTok{\# Create a data frame for Ganado\textquotesingle{}s subsidiaries}
\FunctionTok{library}\NormalTok{(knitr)}

\NormalTok{subsidiaries }\OtherTok{\textless{}{-}} \FunctionTok{data.frame}\NormalTok{(}
  \AttributeTok{Subsidiary =} \FunctionTok{c}\NormalTok{(}\StringTok{"US Parent"}\NormalTok{, }\StringTok{"Europe"}\NormalTok{, }\StringTok{"China"}\NormalTok{),}
  \AttributeTok{Currency =} \FunctionTok{c}\NormalTok{(}\StringTok{"USD"}\NormalTok{, }\StringTok{"EUR"}\NormalTok{, }\StringTok{"CNY"}\NormalTok{),}
  \AttributeTok{Sales\_Local =} \FunctionTok{c}\NormalTok{(}\FloatTok{300.0}\NormalTok{, }\FloatTok{120.0}\NormalTok{, }\FloatTok{600.0}\NormalTok{),}
  \AttributeTok{Earnings\_Local =} \FunctionTok{c}\NormalTok{(}\FloatTok{28.6}\NormalTok{, }\FloatTok{10.5}\NormalTok{, }\FloatTok{71.4}\NormalTok{),}
  \AttributeTok{Exchange\_Rate =} \FunctionTok{c}\NormalTok{(}\FloatTok{1.0000}\NormalTok{, }\FloatTok{1.1200}\NormalTok{, }\FloatTok{0.1515}\NormalTok{),}
  \AttributeTok{Rate\_Description =} \FunctionTok{c}\NormalTok{(}\StringTok{"USD 1.00 = USD 1.00"}\NormalTok{, }
                       \StringTok{"USD 1.12 = EUR 1.00"}\NormalTok{, }
                       \StringTok{"CNY 6.60 = USD 1.00"}\NormalTok{)}
\NormalTok{)}

\CommentTok{\# Calculate USD equivalents}
\NormalTok{subsidiaries}\SpecialCharTok{$}\NormalTok{Sales\_USD }\OtherTok{\textless{}{-}}\NormalTok{ subsidiaries}\SpecialCharTok{$}\NormalTok{Sales\_Local }\SpecialCharTok{*}\NormalTok{ subsidiaries}\SpecialCharTok{$}\NormalTok{Exchange\_Rate}
\NormalTok{subsidiaries}\SpecialCharTok{$}\NormalTok{Earnings\_USD }\OtherTok{\textless{}{-}}\NormalTok{ subsidiaries}\SpecialCharTok{$}\NormalTok{Earnings\_Local }\SpecialCharTok{*}\NormalTok{ subsidiaries}\SpecialCharTok{$}\NormalTok{Exchange\_Rate}

\CommentTok{\# Display the consolidation}
\FunctionTok{kable}\NormalTok{(subsidiaries, }
      \AttributeTok{caption =} \StringTok{"Ganado Corporation {-} Consolidated Results"}\NormalTok{,}
      \AttributeTok{digits =} \DecValTok{2}\NormalTok{)}
\end{Highlighting}
\end{Shaded}

\begin{longtable}[]{@{}
  >{\raggedright\arraybackslash}p{(\linewidth - 14\tabcolsep) * \real{0.1058}}
  >{\raggedright\arraybackslash}p{(\linewidth - 14\tabcolsep) * \real{0.0865}}
  >{\raggedleft\arraybackslash}p{(\linewidth - 14\tabcolsep) * \real{0.1154}}
  >{\raggedleft\arraybackslash}p{(\linewidth - 14\tabcolsep) * \real{0.1442}}
  >{\raggedleft\arraybackslash}p{(\linewidth - 14\tabcolsep) * \real{0.1346}}
  >{\raggedright\arraybackslash}p{(\linewidth - 14\tabcolsep) * \real{0.1923}}
  >{\raggedleft\arraybackslash}p{(\linewidth - 14\tabcolsep) * \real{0.0962}}
  >{\raggedleft\arraybackslash}p{(\linewidth - 14\tabcolsep) * \real{0.1250}}@{}}
\caption{Ganado Corporation - Consolidated Results}\tabularnewline
\toprule\noalign{}
\begin{minipage}[b]{\linewidth}\raggedright
Subsidiary
\end{minipage} & \begin{minipage}[b]{\linewidth}\raggedright
Currency
\end{minipage} & \begin{minipage}[b]{\linewidth}\raggedleft
Sales\_Local
\end{minipage} & \begin{minipage}[b]{\linewidth}\raggedleft
Earnings\_Local
\end{minipage} & \begin{minipage}[b]{\linewidth}\raggedleft
Exchange\_Rate
\end{minipage} & \begin{minipage}[b]{\linewidth}\raggedright
Rate\_Description
\end{minipage} & \begin{minipage}[b]{\linewidth}\raggedleft
Sales\_USD
\end{minipage} & \begin{minipage}[b]{\linewidth}\raggedleft
Earnings\_USD
\end{minipage} \\
\midrule\noalign{}
\endfirsthead
\toprule\noalign{}
\begin{minipage}[b]{\linewidth}\raggedright
Subsidiary
\end{minipage} & \begin{minipage}[b]{\linewidth}\raggedright
Currency
\end{minipage} & \begin{minipage}[b]{\linewidth}\raggedleft
Sales\_Local
\end{minipage} & \begin{minipage}[b]{\linewidth}\raggedleft
Earnings\_Local
\end{minipage} & \begin{minipage}[b]{\linewidth}\raggedleft
Exchange\_Rate
\end{minipage} & \begin{minipage}[b]{\linewidth}\raggedright
Rate\_Description
\end{minipage} & \begin{minipage}[b]{\linewidth}\raggedleft
Sales\_USD
\end{minipage} & \begin{minipage}[b]{\linewidth}\raggedleft
Earnings\_USD
\end{minipage} \\
\midrule\noalign{}
\endhead
\bottomrule\noalign{}
\endlastfoot
US Parent & USD & 300 & 28.6 & 1.00 & USD 1.00 = USD 1.00 & 300.0 &
28.60 \\
Europe & EUR & 120 & 10.5 & 1.12 & USD 1.12 = EUR 1.00 & 134.4 &
11.76 \\
China & CNY & 600 & 71.4 & 0.15 & CNY 6.60 = USD 1.00 & 90.9 & 10.82 \\
\end{longtable}

\begin{Shaded}
\begin{Highlighting}[]
\CommentTok{\# Calculate totals}
\FunctionTok{cat}\NormalTok{(}\StringTok{"}\SpecialCharTok{\textbackslash{}n}\StringTok{Consolidated Totals:}\SpecialCharTok{\textbackslash{}n}\StringTok{"}\NormalTok{)}
\end{Highlighting}
\end{Shaded}

\begin{verbatim}

Consolidated Totals:
\end{verbatim}

\begin{Shaded}
\begin{Highlighting}[]
\FunctionTok{cat}\NormalTok{(}\StringTok{"Total Sales (USD millions):"}\NormalTok{, }\FunctionTok{round}\NormalTok{(}\FunctionTok{sum}\NormalTok{(subsidiaries}\SpecialCharTok{$}\NormalTok{Sales\_USD), }\DecValTok{2}\NormalTok{), }\StringTok{"}\SpecialCharTok{\textbackslash{}n}\StringTok{"}\NormalTok{)}
\end{Highlighting}
\end{Shaded}

\begin{verbatim}
Total Sales (USD millions): 525.3 
\end{verbatim}

\begin{Shaded}
\begin{Highlighting}[]
\FunctionTok{cat}\NormalTok{(}\StringTok{"Total Earnings (USD millions):"}\NormalTok{, }\FunctionTok{round}\NormalTok{(}\FunctionTok{sum}\NormalTok{(subsidiaries}\SpecialCharTok{$}\NormalTok{Earnings\_USD), }\DecValTok{2}\NormalTok{), }\StringTok{"}\SpecialCharTok{\textbackslash{}n}\StringTok{"}\NormalTok{)}
\end{Highlighting}
\end{Shaded}

\begin{verbatim}
Total Earnings (USD millions): 51.18 
\end{verbatim}

\textbf{Question:} ``If the euro falls from \$1.12 to \$1.00, what
happens to consolidated earnings, even if European operations don't
change?''

\begin{Shaded}
\begin{Highlighting}[]
\CommentTok{\# Original scenario}
\NormalTok{eur\_earnings\_local }\OtherTok{\textless{}{-}} \FloatTok{10.5}  \CommentTok{\# million EUR}
\NormalTok{exchange\_rate\_original }\OtherTok{\textless{}{-}} \FloatTok{1.12}
\NormalTok{earnings\_usd\_original }\OtherTok{\textless{}{-}}\NormalTok{ eur\_earnings\_local }\SpecialCharTok{*}\NormalTok{ exchange\_rate\_original}

\CommentTok{\# New scenario {-} euro depreciates}
\NormalTok{exchange\_rate\_new }\OtherTok{\textless{}{-}} \FloatTok{1.00}
\NormalTok{earnings\_usd\_new }\OtherTok{\textless{}{-}}\NormalTok{ eur\_earnings\_local }\SpecialCharTok{*}\NormalTok{ exchange\_rate\_new}

\CommentTok{\# Calculate impact}
\NormalTok{impact }\OtherTok{\textless{}{-}}\NormalTok{ earnings\_usd\_new }\SpecialCharTok{{-}}\NormalTok{ earnings\_usd\_original}

\FunctionTok{cat}\NormalTok{(}\StringTok{"European Subsidiary Earnings Analysis:}\SpecialCharTok{\textbackslash{}n}\StringTok{"}\NormalTok{)}
\end{Highlighting}
\end{Shaded}

\begin{verbatim}
European Subsidiary Earnings Analysis:
\end{verbatim}

\begin{Shaded}
\begin{Highlighting}[]
\FunctionTok{cat}\NormalTok{(}\StringTok{"Local currency earnings (EUR): €"}\NormalTok{, eur\_earnings\_local, }\StringTok{"million (unchanged)}\SpecialCharTok{\textbackslash{}n}\StringTok{"}\NormalTok{)}
\end{Highlighting}
\end{Shaded}

\begin{verbatim}
Local currency earnings (EUR): € 10.5 million (unchanged)
\end{verbatim}

\begin{Shaded}
\begin{Highlighting}[]
\FunctionTok{cat}\NormalTok{(}\StringTok{"}\SpecialCharTok{\textbackslash{}n}\StringTok{Original scenario:}\SpecialCharTok{\textbackslash{}n}\StringTok{"}\NormalTok{)}
\end{Highlighting}
\end{Shaded}

\begin{verbatim}

Original scenario:
\end{verbatim}

\begin{Shaded}
\begin{Highlighting}[]
\FunctionTok{cat}\NormalTok{(}\StringTok{"  Exchange rate: $"}\NormalTok{, exchange\_rate\_original, }\StringTok{"/€}\SpecialCharTok{\textbackslash{}n}\StringTok{"}\NormalTok{)}
\end{Highlighting}
\end{Shaded}

\begin{verbatim}
  Exchange rate: $ 1.12 /€
\end{verbatim}

\begin{Shaded}
\begin{Highlighting}[]
\FunctionTok{cat}\NormalTok{(}\StringTok{"  USD earnings: $"}\NormalTok{, }\FunctionTok{round}\NormalTok{(earnings\_usd\_original, }\DecValTok{2}\NormalTok{), }\StringTok{"million}\SpecialCharTok{\textbackslash{}n}\StringTok{"}\NormalTok{)}
\end{Highlighting}
\end{Shaded}

\begin{verbatim}
  USD earnings: $ 11.76 million
\end{verbatim}

\begin{Shaded}
\begin{Highlighting}[]
\FunctionTok{cat}\NormalTok{(}\StringTok{"}\SpecialCharTok{\textbackslash{}n}\StringTok{New scenario (euro depreciation):}\SpecialCharTok{\textbackslash{}n}\StringTok{"}\NormalTok{)}
\end{Highlighting}
\end{Shaded}

\begin{verbatim}

New scenario (euro depreciation):
\end{verbatim}

\begin{Shaded}
\begin{Highlighting}[]
\FunctionTok{cat}\NormalTok{(}\StringTok{"  Exchange rate: $"}\NormalTok{, exchange\_rate\_new, }\StringTok{"/€}\SpecialCharTok{\textbackslash{}n}\StringTok{"}\NormalTok{)}
\end{Highlighting}
\end{Shaded}

\begin{verbatim}
  Exchange rate: $ 1 /€
\end{verbatim}

\begin{Shaded}
\begin{Highlighting}[]
\FunctionTok{cat}\NormalTok{(}\StringTok{"  USD earnings: $"}\NormalTok{, }\FunctionTok{round}\NormalTok{(earnings\_usd\_new, }\DecValTok{2}\NormalTok{), }\StringTok{"million}\SpecialCharTok{\textbackslash{}n}\StringTok{"}\NormalTok{)}
\end{Highlighting}
\end{Shaded}

\begin{verbatim}
  USD earnings: $ 10.5 million
\end{verbatim}

\begin{Shaded}
\begin{Highlighting}[]
\FunctionTok{cat}\NormalTok{(}\StringTok{"}\SpecialCharTok{\textbackslash{}n}\StringTok{Impact on consolidated earnings: $"}\NormalTok{, }\FunctionTok{round}\NormalTok{(impact, }\DecValTok{2}\NormalTok{), }\StringTok{"million}\SpecialCharTok{\textbackslash{}n}\StringTok{"}\NormalTok{)}
\end{Highlighting}
\end{Shaded}

\begin{verbatim}

Impact on consolidated earnings: $ -1.26 million
\end{verbatim}

\begin{Shaded}
\begin{Highlighting}[]
\FunctionTok{cat}\NormalTok{(}\StringTok{"Percentage decline: "}\NormalTok{, }\FunctionTok{round}\NormalTok{((impact}\SpecialCharTok{/}\NormalTok{earnings\_usd\_original)}\SpecialCharTok{*}\DecValTok{100}\NormalTok{, }\DecValTok{1}\NormalTok{), }\StringTok{"\%}\SpecialCharTok{\textbackslash{}n}\StringTok{"}\NormalTok{)}
\end{Highlighting}
\end{Shaded}

\begin{verbatim}
Percentage decline:  -10.7 %
\end{verbatim}

\textbf{Answer:} Reported earnings fall by over \$1 million purely due
to translation!

\subsection{Political Risk}\label{political-risk}

\textbf{Definition:} Risk that government actions will adversely affect
business operations or asset values.

\subsubsection{Types of Political Risk}\label{types-of-political-risk}

\textbf{1. Transfer Risk:}

\begin{itemize}
\tightlist
\item
  Blocked funds (cannot repatriate earnings)
\item
  Example: Argentina imposing currency controls
\end{itemize}

\textbf{2. Expropriation:}

\begin{itemize}
\tightlist
\item
  Government seizure of assets
\item
  May or may not include compensation
\item
  Less common today but still occurs
\end{itemize}

\textbf{3. Operational Restrictions:}

\begin{itemize}
\tightlist
\item
  Local content requirements
\item
  Technology transfer mandates
\item
  Forced joint ventures
\item
  Price controls
\end{itemize}

\textbf{4. Regulatory Changes:}

\begin{itemize}
\tightlist
\item
  Tax law changes
\item
  Labor law changes
\item
  Environmental regulation
\end{itemize}

\textbf{Recent Examples:}

\begin{itemize}
\tightlist
\item
  Russia's actions in Ukraine affecting Western companies
\item
  China's regulatory crackdowns on tech companies
\item
  India's changing tax treatment of foreign firms
\end{itemize}

\textbf{Key Point:} Political risk is hard to quantify, difficult to
hedge, but absolutely must be considered in international investment
decisions.

\subsection{Modifications to Financial
Theory}\label{modifications-to-financial-theory}

Core financial theories remain valid but require adaptation:

\subsubsection{Cost of Capital}\label{cost-of-capital}

\begin{itemize}
\tightlist
\item
  Must consider multiple country risk premiums
\item
  Currency risk affects required returns
\item
  Access to global capital markets can LOWER cost of capital
\end{itemize}

\subsubsection{Capital Budgeting}\label{capital-budgeting}

\begin{itemize}
\tightlist
\item
  Project cash flows may be in foreign currency
\item
  Repatriation restrictions must be considered
\item
  Country risk must be incorporated
\item
  Multiple tax jurisdictions complicate analysis
\end{itemize}

\subsubsection{Working Capital
Management}\label{working-capital-management}

\begin{itemize}
\tightlist
\item
  Multiple currencies to manage
\item
  International cash pooling opportunities
\item
  More complex credit analysis for foreign customers
\end{itemize}

\subsubsection{Capital Structure}\label{capital-structure}

\begin{itemize}
\tightlist
\item
  Access to foreign debt markets
\item
  Currency matching strategies (financing in same currency as revenues)
\item
  Different optimal leverage ratios in different countries
\end{itemize}

\subsection{Modified Financial
Instruments}\label{modified-financial-instruments}

International finance has adapted standard instruments:

\subsubsection{Currency Derivatives}\label{currency-derivatives}

\begin{itemize}
\tightlist
\item
  Forward contracts
\item
  Futures
\item
  Options on currencies
\item
  Currency swaps
\end{itemize}

\subsubsection{Interest Rate
Instruments}\label{interest-rate-instruments}

\begin{itemize}
\tightlist
\item
  Cross-currency interest rate swaps
\item
  International bonds (Eurobonds, foreign bonds)
\end{itemize}

\subsubsection{Trade Finance
Instruments}\label{trade-finance-instruments}

\begin{itemize}
\tightlist
\item
  Letters of credit
\item
  Banker's acceptances
\item
  Documentary collections
\end{itemize}

\textbf{Preview:} We'll study many of these in detail in later chapters.

\begin{center}\rule{0.5\linewidth}{0.5pt}\end{center}

\section{The Globalization Process}\label{the-globalization-process}

\subsection{Introducing Ganado
Corporation}\label{introducing-ganado-corporation}

\textbf{Ganado is our textbook's case company we'll follow throughout
the semester.}

\subsubsection{Background}\label{background}

\begin{itemize}
\tightlist
\item
  Founded 1948 in Los Angeles
\item
  Originally family-owned telecommunications equipment manufacturer
\item
  Went public (IPO) in 1988 - became a publicly traded company
\item
  Now faces decisions about global expansion
\end{itemize}

\textbf{Why This Matters:} Ganado's journey mirrors what thousands of
real companies experience as they globalize.

\subsection{Phase 1: Domestic to International
Trade}\label{phase-1-domestic-to-international-trade}

\subsubsection{Domestic Phase
Characteristics}\label{domestic-phase-characteristics}

\begin{itemize}
\tightlist
\item
  All suppliers are domestic (US)
\item
  All customers are domestic (US)
\item
  All transactions in US dollars
\item
  Subject only to US laws and regulations
\item
  Credit assessment under familiar US practices
\end{itemize}

\textbf{Simple and low-risk,} but \textbf{limited growth potential.}

\subsubsection{Catalyst for Change}\label{catalyst-for-change}

NAFTA (North American Free Trade Agreement) created opportunities for
trade with Mexico and Canada.

\subsubsection{International Trade
Phase}\label{international-trade-phase}

\begin{itemize}
\tightlist
\item
  BEGIN importing from Mexican suppliers
\item
  BEGIN exporting to Canadian customers
\item
  NOW dealing with multiple currencies (USD, MXN, CAD)
\end{itemize}

\subsubsection{Two New Major Challenges}\label{two-new-major-challenges}

\textbf{1. Foreign Exchange Risk:}

\begin{itemize}
\tightlist
\item
  Which currency to quote prices in?
\item
  Which currency to accept payment in?
\item
  Which currency to pay suppliers in?
\item
  What if exchange rates change between agreement and payment?
\end{itemize}

\textbf{Example:}

\begin{Shaded}
\begin{Highlighting}[]
\CommentTok{\# Ganado quotes Canadian customer}
\NormalTok{quoted\_price\_cad }\OtherTok{\textless{}{-}} \DecValTok{1000}
\NormalTok{exchange\_rate\_at\_quote }\OtherTok{\textless{}{-}} \FloatTok{1.25}  \CommentTok{\# CAD per USD}
\NormalTok{expected\_usd }\OtherTok{\textless{}{-}}\NormalTok{ quoted\_price\_cad }\SpecialCharTok{/}\NormalTok{ exchange\_rate\_at\_quote}

\CommentTok{\# By payment time, CAD depreciates}
\NormalTok{exchange\_rate\_at\_payment }\OtherTok{\textless{}{-}} \FloatTok{1.30}
\NormalTok{actual\_usd }\OtherTok{\textless{}{-}}\NormalTok{ quoted\_price\_cad }\SpecialCharTok{/}\NormalTok{ exchange\_rate\_at\_payment}

\NormalTok{loss }\OtherTok{\textless{}{-}}\NormalTok{ actual\_usd }\SpecialCharTok{{-}}\NormalTok{ expected\_usd}

\FunctionTok{cat}\NormalTok{(}\StringTok{"Foreign Exchange Risk Example:}\SpecialCharTok{\textbackslash{}n}\StringTok{"}\NormalTok{)}
\end{Highlighting}
\end{Shaded}

\begin{verbatim}
Foreign Exchange Risk Example:
\end{verbatim}

\begin{Shaded}
\begin{Highlighting}[]
\FunctionTok{cat}\NormalTok{(}\StringTok{"Quoted price: CAD"}\NormalTok{, quoted\_price\_cad, }\StringTok{"}\SpecialCharTok{\textbackslash{}n}\StringTok{"}\NormalTok{)}
\end{Highlighting}
\end{Shaded}

\begin{verbatim}
Quoted price: CAD 1000 
\end{verbatim}

\begin{Shaded}
\begin{Highlighting}[]
\FunctionTok{cat}\NormalTok{(}\StringTok{"Exchange rate at quote: CAD"}\NormalTok{, exchange\_rate\_at\_quote, }\StringTok{"= USD 1.00}\SpecialCharTok{\textbackslash{}n}\StringTok{"}\NormalTok{)}
\end{Highlighting}
\end{Shaded}

\begin{verbatim}
Exchange rate at quote: CAD 1.25 = USD 1.00
\end{verbatim}

\begin{Shaded}
\begin{Highlighting}[]
\FunctionTok{cat}\NormalTok{(}\StringTok{"Expected USD receipt: $"}\NormalTok{, }\FunctionTok{round}\NormalTok{(expected\_usd, }\DecValTok{2}\NormalTok{), }\StringTok{"}\SpecialCharTok{\textbackslash{}n\textbackslash{}n}\StringTok{"}\NormalTok{)}
\end{Highlighting}
\end{Shaded}

\begin{verbatim}
Expected USD receipt: $ 800 
\end{verbatim}

\begin{Shaded}
\begin{Highlighting}[]
\FunctionTok{cat}\NormalTok{(}\StringTok{"Exchange rate at payment: CAD"}\NormalTok{, exchange\_rate\_at\_payment, }\StringTok{"= USD 1.00}\SpecialCharTok{\textbackslash{}n}\StringTok{"}\NormalTok{)}
\end{Highlighting}
\end{Shaded}

\begin{verbatim}
Exchange rate at payment: CAD 1.3 = USD 1.00
\end{verbatim}

\begin{Shaded}
\begin{Highlighting}[]
\FunctionTok{cat}\NormalTok{(}\StringTok{"Actual USD receipt: $"}\NormalTok{, }\FunctionTok{round}\NormalTok{(actual\_usd, }\DecValTok{2}\NormalTok{), }\StringTok{"}\SpecialCharTok{\textbackslash{}n\textbackslash{}n}\StringTok{"}\NormalTok{)}
\end{Highlighting}
\end{Shaded}

\begin{verbatim}
Actual USD receipt: $ 769.23 
\end{verbatim}

\begin{Shaded}
\begin{Highlighting}[]
\FunctionTok{cat}\NormalTok{(}\StringTok{"Loss due to currency movement: $"}\NormalTok{, }\FunctionTok{round}\NormalTok{(loss, }\DecValTok{2}\NormalTok{), }\StringTok{"}\SpecialCharTok{\textbackslash{}n}\StringTok{"}\NormalTok{)}
\end{Highlighting}
\end{Shaded}

\begin{verbatim}
Loss due to currency movement: $ -30.77 
\end{verbatim}

\textbf{2. Credit Risk Assessment:}

\begin{itemize}
\tightlist
\item
  How creditworthy is Mexican supplier?
\item
  Will Canadian customer actually pay?
\item
  Different business practices
\item
  Different legal systems
\item
  Different bankruptcy laws
\item
  Less available information
\end{itemize}

\textbf{Key Point:} Even in this relatively simple international trade
phase, financial management complexity increases dramatically compared
to purely domestic operations.

\subsection{Phase 2: International Trade to Multinational
Enterprise}\label{phase-2-international-trade-to-multinational-enterprise}

\textbf{Critical Decision Point:} At some point, successful
international trading leads companies to consider foreign direct
investment (FDI).

\subsubsection{Why Make This Leap?}\label{why-make-this-leap}

\textbf{Strategic Motivations (Five Categories):}

\textbf{1. Market Seekers:}

\begin{itemize}
\tightlist
\item
  Produce in foreign markets to serve local demand
\item
  Overcome trade barriers
\item
  Get closer to customers
\item
  Example: US auto manufacturers in Europe
\end{itemize}

\textbf{2. Raw Material Seekers:}

\begin{itemize}
\tightlist
\item
  Access natural resources where they're located
\item
  Mining, oil, forestry, agricultural products
\item
  Example: Oil companies drilling in Middle East
\end{itemize}

\textbf{3. Production Efficiency Seekers:}

\begin{itemize}
\tightlist
\item
  Take advantage of lower-cost factors of production
\item
  Especially labor
\item
  Example: Electronics assembly in Malaysia, Vietnam
\end{itemize}

\textbf{4. Knowledge Seekers:}

\begin{itemize}
\tightlist
\item
  Access technology, expertise, innovation
\item
  Example: Japanese and German companies buying US tech firms
\end{itemize}

\textbf{5. Political Safety Seekers:}

\begin{itemize}
\tightlist
\item
  Diversify political risk
\item
  Example: Hong Kong firms investing in Canada, Australia before 1997
  Chinese takeover
\end{itemize}

\textbf{Important Note:} These are not mutually exclusive. One
investment may serve multiple purposes.

\subsection{The Foreign Direct Investment
Spectrum}\label{the-foreign-direct-investment-spectrum}

\subsubsection{Increasing Foreign Presence (Low to
High)}\label{increasing-foreign-presence-low-to-high}

\textbf{1. Exporting from Home Production}

\begin{itemize}
\tightlist
\item
  Lowest commitment
\item
  Maintain control
\item
  Limited foreign presence
\end{itemize}

\textbf{2. Licensing to Foreign Firms}

\begin{itemize}
\tightlist
\item
  Foreign company produces under your brand/technology
\item
  Receive royalties
\item
  Low capital investment
\item
  Risk: Creating future competitors, loss of quality control
\end{itemize}

\textbf{3. Joint Venture}

\begin{itemize}
\tightlist
\item
  Partner with local firm (often 50-50 ownership)
\item
  Share investment, share risks, share rewards
\item
  Common in countries requiring local partners
\item
  Challenge: Potential conflicts with partner
\end{itemize}

\textbf{4. Wholly-Owned Subsidiary - Greenfield}

\begin{itemize}
\tightlist
\item
  Build new facilities from scratch (``greenfield'')
\item
  Full control
\item
  High investment
\item
  Slow to establish
\end{itemize}

\textbf{5. Wholly-Owned Subsidiary - Acquisition}

\begin{itemize}
\tightlist
\item
  Buy existing foreign company
\item
  Fastest market entry
\item
  Immediate operations
\item
  Integration challenges
\item
  Potentially very expensive
\end{itemize}

\textbf{Key Insight:} As you move down this list:

\begin{itemize}
\tightlist
\item
  \textbf{Foreign presence increases} (more physical assets, more
  employees abroad)
\item
  \textbf{Managerial intensity increases} (more complex to manage)
\item
  \textbf{Capital at risk increases} (more investment)
\item
  \textbf{Potential returns increase} (greater opportunities)
\end{itemize}

\subsection{Consolidation Challenges}\label{consolidation-challenges}

This is a critical practical issue. Once Ganado has foreign
subsidiaries:

\begin{itemize}
\tightlist
\item
  European subsidiary keeping books in euros
\item
  Chinese subsidiary keeping books in yuan
\item
  US parent keeping books in dollars
\end{itemize}

\subsubsection{The Problem}\label{the-problem}

Ganado must periodically (quarterly, annually) produce consolidated
financial statements in US dollars for:

\begin{itemize}
\tightlist
\item
  Shareholders
\item
  SEC (Securities and Exchange Commission) regulations
\item
  Tax purposes
\item
  Performance evaluation
\end{itemize}

\subsubsection{The Process}\label{the-process}

\begin{enumerate}
\def\labelenumi{\arabic{enumi}.}
\tightlist
\item
  Each subsidiary reports in local currency
\item
  Convert to common currency (USD) using appropriate exchange rates
\item
  Sum to create consolidated totals
\end{enumerate}

\textbf{Critical Issue:} Exchange rates change constantly. From one
quarter to the next:

\begin{itemize}
\tightlist
\item
  Foreign subsidiary performance may be stable
\item
  BUT exchange rate changes make it look better or worse
\item
  This creates ``translation exposure'' or ``accounting exposure''
\end{itemize}

\begin{center}\rule{0.5\linewidth}{0.5pt}\end{center}

\section{Limits to Financial Globalization (8
minutes)}\label{limits-to-financial-globalization-8-minutes}

\subsection{Not All Globalization is Inevitable or
Beneficial}\label{not-all-globalization-is-inevitable-or-beneficial}

Classical economic theory suggests capital should flow freely to its
most productive uses globally. \textbf{Reality is more complex.}

\subsection{The ``Twin Agency
Problems''}\label{the-twin-agency-problems}

\textbf{Agency Problem Definition:} When someone managing assets (agent)
has different interests than the asset owners (principals)

\subsubsection{First Agency Problem - Corporate
Insiders}\label{first-agency-problem---corporate-insiders}

\begin{itemize}
\tightlist
\item
  Do managers/executives pursue firm value maximization?
\item
  OR do they pursue personal wealth/power?
\end{itemize}

\textbf{Examples of Value-Destroying Insider Actions:}

\begin{itemize}
\tightlist
\item
  Excessive compensation
\item
  Empire building (growth for prestige, not returns)
\item
  Resistance to beneficial takeovers
\item
  Related-party transactions that benefit insiders
\end{itemize}

\subsubsection{Second Agency Problem - Government/Sovereign
Insiders}\label{second-agency-problem---governmentsovereign-insiders}

\begin{itemize}
\tightlist
\item
  Do political leaders pursue national economic welfare?
\item
  OR do they pursue personal power/wealth?
\end{itemize}

\textbf{Examples:}

\begin{itemize}
\tightlist
\item
  Corruption
\item
  Expropriation
\item
  Capital controls that harm economy but protect political power
\item
  Regulatory capture
\end{itemize}

\subsection{Consequences for Global Capital
Flows}\label{consequences-for-global-capital-flows}

\textbf{IF corporate and government insiders pursue value maximization:}

→ Capital flows to these opportunities\\
→ Financial globalization INCREASES\\
→ Benefits widely shared

\textbf{IF insiders pursue personal agendas:}

→ Capital avoids these situations\\
→ Financial globalization LIMITED\\
→ Creates ``winners'' and ``losers''\\
→ Segmented markets

\subsubsection{Real-World Examples}\label{real-world-examples}

\textbf{Positive Examples (Attracting Capital):}

\begin{itemize}
\tightlist
\item
  Singapore: Strong rule of law, low corruption, excellent governance
\item
  Result: Major financial center despite small size
\end{itemize}

\textbf{Negative Examples (Capital Flight):}

\begin{itemize}
\tightlist
\item
  Venezuela: Expropriation, corruption, poor governance
\item
  Result: Economic collapse, capital flees
\end{itemize}

\subsection{Contemporary Challenges}\label{contemporary-challenges}

\subsubsection{Three Current Issues}\label{three-current-issues}

\textbf{1. Inequality:}

\begin{itemize}
\tightlist
\item
  Globalization benefits not evenly distributed
\item
  Within-country inequality increasing
\item
  ``Left behind'' populations creating political backlash
\end{itemize}

\textbf{2. Regulatory Arbitrage:}

\begin{itemize}
\tightlist
\item
  Companies structure operations to minimize taxes
\item
  ``Tax havens'' and profit shifting
\item
  Creates political tensions
\end{itemize}

\textbf{3. National Security vs.~Economic Efficiency:}

\begin{itemize}
\tightlist
\item
  Growing restrictions on foreign investment in ``strategic'' sectors
\item
  Technology transfer concerns
\item
  Supply chain security concerns
\end{itemize}

\textbf{Example:} US restrictions on Chinese investment in American
technology companies, even when economically beneficial.

\subsection{The Path Forward}\label{the-path-forward}

\begin{quote}
``Welcome to the future. This will be a constant struggle. We need
leadership, citizenship, and dialogue.''\\
- Donald Lessard
\end{quote}

\textbf{Key Takeaway:} Financial globalization is neither inevitable nor
uniformly beneficial. Success requires:

\begin{itemize}
\tightlist
\item
  Strong institutions
\item
  Good governance\\
\item
  Ethical leadership
\item
  Informed citizenry
\item
  Ongoing dialogue and adaptation
\end{itemize}

\begin{center}\rule{0.5\linewidth}{0.5pt}\end{center}

\section{Conclusion and Synthesis (5
minutes)}\label{conclusion-and-synthesis-5-minutes}

\subsection{Connecting the Dots}\label{connecting-the-dots}

Let me bring together today's major themes:

\subsubsection{The Global Financial
Marketplace}\label{the-global-financial-marketplace-1}

\begin{itemize}
\tightlist
\item
  Complex ecosystem of securities, institutions, and currency linkages
\item
  Larger and more interconnected than ever
\item
  Creates both opportunities and risks
\end{itemize}

\subsubsection{Comparative Advantage}\label{comparative-advantage}

\begin{itemize}
\tightlist
\item
  Still explains why international trade benefits countries
\item
  Modern version focuses on services, skills, technology
\item
  Dynamic, not static - shifts over time
\end{itemize}

\subsubsection{What Makes International Finance
Unique}\label{what-makes-international-finance-unique}

\begin{itemize}
\tightlist
\item
  Foreign exchange risk
\item
  Political risk
\item
  Cultural and institutional differences
\item
  Corporate governance variations
\item
  Required modifications to financial theories and instruments
\end{itemize}

\subsubsection{The Globalization
Journey}\label{the-globalization-journey}

\begin{itemize}
\tightlist
\item
  Progression from domestic → international trade → multinational
\item
  Each phase brings new opportunities and new complexities
\item
  Not all firms should fully globalize
\item
  Not all globalization creates value
\end{itemize}

\subsubsection{Limits and Challenges}\label{limits-and-challenges}

\begin{itemize}
\tightlist
\item
  Agency problems at corporate and sovereign levels
\item
  Potential for value destruction
\item
  Ongoing tension between efficiency and other goals (security, equity,
  stability)
\end{itemize}

\subsection{Looking Ahead}\label{looking-ahead}

\textbf{Next chapters will build on this foundation:}

\begin{itemize}
\tightlist
\item
  Chapter 2: International Monetary System - How currencies interact,
  exchange rate regimes
\item
  Chapter 3: Balance of Payments - How we track international flows
\item
  Chapter 4: Corporate Governance in depth - Different models worldwide
\item
  Later: Foreign exchange markets, currency derivatives, international
  investment, multinational capital budgeting
\end{itemize}

\subsection{Why This Matters to You}\label{why-this-matters-to-you}

\subsubsection{Career Relevance}\label{career-relevance}

\begin{itemize}
\tightlist
\item
  Almost every major company operates internationally
\item
  Understanding these concepts makes you more valuable
\item
  Finance, marketing, operations, strategy all require international
  awareness
\end{itemize}

\subsubsection{Citizenship}\label{citizenship}

\begin{itemize}
\tightlist
\item
  Trade policy
\item
  Tax policy
\item
  Immigration
\item
  National security
\end{itemize}

All involve international finance concepts.

\subsubsection{Personal Finance}\label{personal-finance}

\begin{itemize}
\tightlist
\item
  International diversification
\item
  Currency considerations
\item
  Understanding global markets
\end{itemize}

\subsection{Practical Advice for Success in This
Course}\label{practical-advice-for-success-in-this-course}

\subsubsection{Stay Current}\label{stay-current}

\begin{itemize}
\tightlist
\item
  Read \emph{Financial Times}, \emph{Wall Street Journal}, \emph{The
  Economist}
\item
  Follow exchange rates
\item
  Track international business news
\end{itemize}

\subsubsection{Think Critically}\label{think-critically}

\begin{itemize}
\tightlist
\item
  Don't just memorize formulas
\item
  Understand the ``why'' behind concepts
\item
  Question assumptions
\end{itemize}

\subsubsection{Connect Concepts}\label{connect-concepts}

\begin{itemize}
\tightlist
\item
  International finance is integrative
\item
  Links to economics, politics, strategy, culture
\item
  Make connections across chapters
\end{itemize}

\subsubsection{Use Real Examples}\label{use-real-examples}

\begin{itemize}
\tightlist
\item
  Apply concepts to companies you know
\item
  Follow a multinational company throughout semester
\item
  Analyze current events through international finance lens
\end{itemize}

\begin{center}\rule{0.5\linewidth}{0.5pt}\end{center}

\section{Questions and Discussion (5-10
minutes)}\label{questions-and-discussion-5-10-minutes}

\subsection{Check for Understanding}\label{check-for-understanding}

\textbf{Key Questions to Pose:}

\begin{enumerate}
\def\labelenumi{\arabic{enumi}.}
\item
  ``Can someone explain in their own words what makes the eurocurrency
  market different from domestic banking?''
\item
  ``Why might a country with absolute advantage in producing everything
  still benefit from trade?''
\item
  ``What's the single biggest risk factor that distinguishes
  international from domestic finance?''
\item
  ``If the dollar strengthens against all foreign currencies, what
  generally happens to US multinationals' consolidated earnings, all
  else equal?''
\end{enumerate}

\subsection{Preview Next Class}\label{preview-next-class}

\textbf{Next session we'll dive into:}

\begin{itemize}
\tightlist
\item
  The International Monetary System (Chapter 2)
\item
  Fixed vs.~floating exchange rates
\item
  The role of the IMF and World Bank
\item
  Major currency crises in history
\item
  The future of the dollar as reserve currency
\end{itemize}

\textbf{Assignment for Next Class:}

\begin{itemize}
\tightlist
\item
  Read Chapter 2
\item
  Find a current article about exchange rate movements
\item
  Be prepared to discuss why that currency moved
\end{itemize}

\begin{center}\rule{0.5\linewidth}{0.5pt}\end{center}

\section{Final Thoughts (2 minutes)}\label{final-thoughts-2-minutes}

International finance is one of the most dynamic, challenging, and
relevant fields in business today. Every concept we study has real-world
application. Every formula we learn represents actual business
decisions.

The world is more connected than ever, which creates unprecedented
opportunities for those who understand how to navigate the global
financial marketplace. It also creates risks for those who don't.

My goal is to help you develop not just technical competence, but also
strategic insight and critical thinking about international finance.
This first chapter laid the foundation - now we build.

See you next class. Please review Chapter 1, attempt the practice
problems, and don't hesitate to email me with questions.




\end{document}
